\documentclass[twoside,11pt]{article}
\usepackage[slovene]{babel}
\usepackage[utf8]{inputenc}
\usepackage{graphicx}
\usepackage[frame]{matrika}
\usepackage{mathtools}
\usepackage{epstopdf}
\usepackage{units}
\usepackage{url}
\usepackage{amsfonts}
\usepackage{amsmath}


\begin{document}

\MAT{1}{10}{2024}
\naslov{Stohastični abakus}

\avtor{Luka Houška}

\institucija{Fakulteta za matematiko in fiziko \\ Univerza v Ljubljani}

\date{23.10.2023}

\klasifikacija{~} 
\izvlecek{V članku je podrobno predstavljen stohastični abakus, kako algoritem deluje, zakaj deluje in zakaj se ustavi. S pomočjo stohastičnega abakusa so analizirane igre "podajanja evra" na različnih grafih, kar razkrije naravno povezavo s Fibonaccijevimi števili.}
\title{Probabilistic abacus}
\abstract{The article presents stochastic abacus in detail, how the algorithm works, why it works and why it stops. Using the stochastic abacus, pass-the-buck games are analyzed on various graphs, revealing a natural connection with the    Fibonacci numbers.}

\glava
\baselineskip=14.5pt

\smallskip

\section{Uvod}
Igra podajanja evra se začne z $n$ igralci postavljenimi kot vozlišča grafa. En igralec ima v roki evro. Igra se igra izmenično, kjer v vsakem koraku igralec naključno zmaga ali pa poda evro enemu od igralcev na sosednjih vozliščih. Če ima vozlišče trenutnega igralca stopnjo $d$ je vsak od $d + 1$ izidov enako verjeten. Pri igri bomo uporabljali koncept chip firing, katerega je kot sredstvo za poučevanje verjetnosti osnovnošolcem razvil oče stohastičnega abakusa Arthur Engel.
Osnovna ideja chip firinga je preprosta. Če imamo podan graf, si predstavljamo, da na vozlišča postavimo nekaj žetonov. Če je število žetonov na vozlišču $v$ večje ali enako kot stopnja $v$, potem lahko \emph{izstrelimo} $v$ tako, da po vsaki povezavi, ki ima za eno od krajišč vozlišče $v$ pošljemo en žeton do sosednjega vozlišča. Število žetonov v vozlišču $v$ se po izstrelitvi zmanjša za stopnjo $v$, vsakemu od sosednjih vozlišč $v$ pa se število žetonov poveča za ena. 
Koncept chip firing bomo implementirali na \emph{usmerjene grafe}, kjer bo izstrelitev vozlišča $v$ pomenila, da bo po vsaki izhodni povezavi iz $v$ pošljemo en žeton. Natančneje, usmerjeni graf bo prehodni graf za absorbirajočo Markovsko verigo, ki ustreza igri podajanja evra. Naša naloga je, da natančno predstavimo Engelov originalni koncept, ki uporablja chip firing algoritem kot sredstvo za računanje verjetnosti in pričakovanih vrednosti. Naš prvi korak je spoznati Engelov algoritem.

\section{Stohastični abakus}
V začetku 1970-ih je Engel razvil svoj algoritem, ki ga je kasneje poimenoval stohastični abakus, saj premikanje žetonov po povezavah v grafu za namen računanja spominja na tradicionalni abakus.
\indent Predpostavimo, da imamo tako časovno diskretno, absorbirajočo Markovsko verigo s končnim število stanj, da so vse prehodne verjetnosti racionalne. Predstavimo jo z usmerjenim grafom, kjer so stanja vozlišča in usmerjena povezava med vozliščema $u$ in $v$ obstaja natanko tedaj, ko je verjetnost direktnega prehoda iz $u$ v $v$ pozitivna. 
Natančneje, če je $q$ najmanjši skupni imenovalec vseh prehodnih verjetnosti iz stanja $u$ in je verjetnost prehoda iz $u$ v $v$ enaka $p/q$, potem postavimo p usmerjenih povezav iz $u$ v $v$. Če to naredimo za vse sosede $u$, je izhodna stopnja $u$ enaka $q$. Zato se prehodne verjetnosti lahko razberejo iz usmerjenega grafa brez uporabe uteži na povezavah. 
Uporabljali bomo izraza \emph{notranja vozlišča} za minljiva stanja in \emph{končna vozlišča} za absorbirajoča stanja. \\
\indent Ideja algortima je, da začnemo s praznimi vozlišči in dodajamo žetone na začetno stanje, ki ga bomo poimenovali začetno vozlišče. Ko je v začetnem vozlišču dovolj žetonov, ga izstrelimo tako, da po povezavah do sosednjih vozlišč pošljemo žetone v razmerju, ki odražajo prehodne verjetnosti. Ponavljamo postopek dodajanja žetonov v začetno vozlišče in izstreljevanja dokler nima vsak od sosedov dovolj žetonov, da izstrelimo te itn. Predstavljajmo si, da ponavljamo postopek dodajanja žetonov in izstrelitev vozlišč in
da je na neki točki število žetonov na notranjih vozliščih enako začetnemu. Engel je utemeljil, da je v tem trenutku število žetonov v končnih vozliščih proporcionalno njenim verjetnostim, saj premikanje žetonov po grafu upošteva prehodne verjetnosti. %Če smo bolj natančni: če ima končno vozlišče $k$ žetonov in je število vseh dodanih žetonov v začetno vozlišče $c$, potem je verjetnost, da pridemo v to absorbirajoče stanje iz začetnega stanja enaka $k/c$. Ker smo ustvarili periodičen proces; če dodajamo žetone v začetno vozlišče in ponovimo celoten postopek, bo končna razporeditev žetonov v končnih vozliščih enaka, saj more po zakonu o velikih številih biti v razmerju z verjetnostmi. Ker pa je proces periodičen je ena iteracija dovolj, da odkrijemo pravilna razmerja žetonov v končnih vozliščih. \\
\newline 
\indent S tem algoritmom sta dva potencialna problema in Engel je uspel razrešiti oba. Prvič, ni težko pokazati, da je vrstni red izstrelitev vozlišč nepomemben. To pomeni, da je sproščanje sistema, izstrelitev vseh primernih vozlišč, dobro definirano. Drugič, treba je ugotoviti kako prvotno postaviti žetone na notranjih vozliščih, da imamo zagotovljeno, da se bo sistem vrnil v to razporeditev. Izkaže se, da če na začetku vsako notranje vozlišče "kritično naložimo", tj. ima en žeton manj kot ima izhodnih povezav, dosežemo, da se bodo notranja vozlišča vrnila v kritično naloženo stanje po končnem številu dodanih žetonov v začetno vozlišče. Podajmo še dokaz.

\vspace{\baselineskip}

\noindent {\em Dokaz:\/} Zapišimo prehodne verjetnosti kot $p_{ij} = \frac{r_{ij}}{r_{i}},\; r_{ij}, r_{i}\in \mathbb{N}$. Potezo, kjer iz vozlišča $i$ izstrelimo $r_{ij}$ žetonov v vozlišče $j$ za vsak $j$, bomo poimenovali poteza \emph{tipa 1}. Potezo, kjer vozlišču $u$ dodamo žeton bomo poimenovali poteza \emph{tipa 2}. Izberimo še neko minljivo stanje $u$, ki bo naše začetno vozlišče. Začnimo z neko začetno porazdelitvijo žetonov, ki je manjša ali enaka kot kritična. Nadaljujemo z algoritmom dokler se neka porazdelitev $b_{1},b_{2},\ldots,b_{r}$ ne ponovi. Takšna porazdelitev zagotovo obstaja, saj imamo lahko le končno mnogo različnih porazdelitev. Vzemimo to porazdelitev $b_{1},b_{2},\ldots,b_{r}$ za začetno. Lahko predpostavimo, da je $b_{i}<r_{i}$ za $i \neq u$ in $b_{u} = r_{u}$. Izberimo za to porazdelitev zelene žetone. Nato dodajmo rdeče žetone, da razporeditev dopolnimo do kritične. Nato nadaljujemo z enakim zaporedjem potez, vendar uporabljamo samo zelene žetone. Ko se bo naša začetna porazdelitev ponovila, bomo spet imeli kritično razporeditev, saj se rdečih žetonov nismo dotaknili. Poimenujmo ta modificiran postopek \emph{metoda A}. Poimenovali smo ga modificiran, saj nismo upoštevali Engelovih pravil o začetni porazdelitvi žetonov. 
Predpostavimo, da smo v metodi A v vozlišče $u$ dodali $m$ žetonov. Poglejmo zdaj še drugo metodo, ki jo bomo poimenovali \emph{metoda B}. V metodi B bomo začeli s kritično porazdelitvijo žetonov in izvajali algoritem dokler ne dodamo $m$ žetonov in ne moremo narediti več nobene poteze prvega tipa. Pokazali bomo, da obe metodi pripeljeta do enake končne razporeditve. Ker vemo, da je končna razporeditev v metodi A kritična, bo enako veljalo tudi za metodo B. \newline
\indent Naj bo $e_{1},e_{2},\ldots,e_{n}$ zaporedje potez, ki smo jih naredili v metodi A in naj bo $f_{1},f_{2},\ldots,f_{v}$ zaporedje potez v metodi B. Potezi $e_{1}$ in $f_{1}$ sta enaki, saj v obeh primerih izstrelimo $r_{u}$ žetonov iz vozlišča $u$. Z indukcijo bomo dokazali, da se vse poteze $e_{i}, i \in [n] $ pojavijo nekje v zaporedju $\{ f_{j}\} _{j\in [v] }$. Predpostavimo, da smo za poteze $e_{1},e_{2},\ldots,e_{k}$ našli enake poteze $f_{i_1},f_{i_2},\ldots,f_{i_k}$. Poglejmo potezo $e_{k+1}$. Če je to poteza tipa 2, mora obstajati enaka poteza $f_{i_{k+1}}$, saj v obeh metodah naredimo $m$ takšnih potez.
Predpostavimo sedaj, da je $e_{k+1}$ poteza tipa 1, recimo, da izstrelimo $r_{t}$ žetonov iz vozlišča $t$. Če obstaja taka poteza $f_{i_{k+1}}$, ki se pojavi poleg $f_{i_1},f_{i_2},\ldots,f_{i_k}$ in preden se vse te izvedejo, potem lahko to potezo izberemo kot ujemajočo se s $e_{k+1}$. Predpostavimo sedaj, da take poteze ni. Poteza $e_{k+1}$ je bila omogočena z zaporedjem potez $e_{1},e_{2},\ldots,e_{k}$. Torej more biti omogočena tudi z zaporedjem potez $f_{i_1},f_{i_2},\ldots,f_{i_k}$. Druge poteze, ki bi se lahko zgodile medtem ko se izvajajo te poteze, lahko kvečjemu povišajo število žetonov v stanju $t$. 
Zato ponovno, imamo potezo, ki jo lahko enačimo z $e_{k+1}$ in indukcijski korak je dokazan. Analogno dokažemo, da so vsi $f_{j}, j \in [v] $ vsebovani v $\{e_{i} \}_{i\in [n] }$. Zato sta ti dve množici potez enaki, kar implicira, da metodi A in B vodita do enake končne porazdelitve, kar smo hoteli dokazati.
\hfill \QED

\vspace{\baselineskip}

Algoritem vrne dva pomembna rezultata, $n_{uj}$, pričakovano število obiskov stanja $j$, če proces začnemo v stanju $u$, in $b_{uj}$, verjetnost, da se proces konča v absorbirajočem stanju $j$, če se začne v stanju $u$. V naslednji trditvi bomo navedli formuli za ta dva rezultata.

\begin{trditev}\label{formule}
    Naj bo $u$ začetno stanje. Dalje naj bo $w_{uj}$ skupno število žetonov, ki so bili izstreljeni iz minljivega stanja $j$ in naj bo $v_{uk}$ število žetonov, ki so med algoritmom prispeli v absorbirajoče stanje $k$. Naj bo $v_{u} = \sum_{k} v_{uk}$. Potem velja: 
    \begin{enumerate}
        \item[a)] $n_{uj} = \frac{w_{uj}}{v_{u}}$
        \item[b)] $b_{uj} = \frac{v_{uj}}{v_{u}}$
    \end{enumerate}

\end{trditev}


\noindent
Preden začnemo z dokazom vpeljimo še nekaj oznak. $S = \{1,2,\ldots,n\}$ naj bo množica stanj in $I = \{1,2,\ldots,r\}$ naj bo množica minljivih stanj. $P = (p_{ij})_{i,j=[n]}$ naj bo prehodna matrika markovske verige. Prehodno matriko potem lahko zapišemo kot 
$ P =
\begin{bmatrix} 
    Q & R \\
    0 & I 
\end{bmatrix}
$
    , kjer $Q$ predstavlja prehodne verjetnosti le med minljivimi stanji, $R$ predstavlja prehodne verjetnosti iz minljivih stanj v absorbirajoča, $I$ pa je identična matrika, saj veriga ostane v absorbirajočih stanjih. Z $N = (I-Q)^{-1}$ označimo fundamentalno matriko markovske verige, njeni elementi $n_{ij}$  pa predstavljajo pričakovano število obiskov stanja $j\in S$, če začnemo v stanju $i\in I$. V dokazu točke a) bomo potrebovali še sledeči lemi.  
\smallskip

\begin{lema}\label{markovske}
    Naj bo $x$ minljivo stanje v markovski verigi s končnim številom stanj. Naj bo $p_{y,x}(n)$ verjetnost prehoda iz stanja $y$ v stanje $x$ v $n$ korakih. Potem velja $\lim_{n \to \infty} p_{y,x}(n) = 0$.
\end{lema}

\begin{lema}\label{matrike}
    Naj bo $A$ končna matrika in naj bo njen spektralni radij $\rho(A)<1$. Potem je matrika $(I-A)$ obrnljiva.
\end{lema}


\noindent Dokažimo sedaj trditev \ref{formule}. \newline
\smallskip

\noindent {\em Dokaz \ref{formule}:\/}
\begin{enumerate}
    \item[a)] Od prej vemo, da se nam bo v algoritmu, ki ga začnemo s poljubno začetno porazdelitvijo, neka porazdelitev $b_{1},b_{2},\ldots,b_{r}$ ponovila, saj je število različnih porazdelitev končno. Začnimo sedaj algoritem znova, le da je tokrat začetna porazdelitev enaka $b_{1},b_{2},\ldots,b_{r}$. Ker sta začetni in končni porazdelitvi enaki, mora za vsako stanje veljati, da je število žetonov, ki so vanj prispeli, enako številu žetonov, ki so stanje zapustili. Zato je 
$$w_{uu} = v_{u} + \sum_{k} w_{uk}p_{ku},$$ saj je število žetonov, ki so prispeli v začetno vozlišče $u$ enako številu žetonov, ki smo ga tekom algoritma dodali, torej $v_{u}$, zraven moramo pa dodati še žetone, ki so tekom algoritma vanj prispeli iz stanj, torej za vsak $k$ prištejemo produkt $w_{uk}$, števila izstreljenih žetonov iz $k$, in $p_{ku}$, ki predstavlja delež žetonov, ki pri vsaki izstrelitvi stanja $k$ gredo v $u$. 
Podobno premislimo, da velja
$$w_{ui} = \sum_{k} w_{uk}p_{ki}, \;i\neq u.$$
Naj bo $\overline{w}_{uj} = \frac{w_{uj}}{v_{u}}$. Potem je 
\[
    \begin{split}
        \overline{w}_{uu} &= 1 + \sum_{k} \overline{w}_{uk}p_{ku} \\
        \overline{w}_{ui} &= \sum_{k} \overline{w}_{uk}p_{ki}, \; i \neq u
    \end{split}
\]

\noindent Če to naredimo za vse izbire začetnega stanja $u$ dobimo matrično enačbo
$$\overline{W} = I + \overline{W}Q$$
oziroma
$$\overline{W} = (I - Q)^{-1} = N.$$
Preveriti moramo še, da je matrika $(I-Q)$ obrnljiva. Pokažimo, da za spektralni radij matrike $Q$ velja $\rho(Q) < 1$. Spomnimo se, da element $p_{ij}$ v matriki $Q^n$ predstavlja verjetnost prehoda iz stanja $i$ v stanje $j$ v $n$ korakih. Ker so v matriki $Q$ le minljiva stanja, po lemi \ref{markovske} sledi, da gre $Q^n \rightarrow 0$, ko pošljemo $n\rightarrow \infty$. Naj bo $\lambda$ lastna vrednost matrike $Q$ in $z$ njen pripadajoč lastni vektor. 
Spomnimo se, da je tudi $\widetilde{v} = \frac{v}{\left\lVert v\right\rVert }$ lastni vektor za lastno vrednost $\lambda$ in da je $\left\lVert \widetilde{v}\right\rVert = 1$. Potem iz $Q\widetilde{v} = \lambda\widetilde{v}$ oziroma $Q^n\widetilde{v} = \lambda^n\widetilde{v}$ sledi
\[
    \left\lVert Q^n\widetilde{v} \right\rVert = \left\lVert \lambda^n\widetilde{v}\right\rVert = \left\lvert \lambda^n\right\rvert\cdot \left\lVert \widetilde{v}\right\rVert = \left\lvert \lambda^n\right\rvert = \left\lvert \lambda\right\rvert^n 
\]
Po drugi strani vemo, da je $\left\lVert Q^n\widetilde{v} \right\rVert\leq \left\lVert Q^n \right\rVert \cdot \left\lVert \widetilde{v} \right\rVert = \left\lVert Q^n \right\rVert$
Torej velja
$$ \left\lVert Q^n \right\rVert \geq \left\lvert \lambda\right\rvert^n$$
Ker vemo, da gre $Q^n \rightarrow 0$, ko gre $n \rightarrow \infty$, gre torej leva stran zadnje neenačbe proti 0 in zato mora iti tudi $\left\lvert \lambda \right\rvert^n \rightarrow 0$. To pa se bo zgodilo natanko tedaj, ko bo $\left\lvert \lambda \right\rvert < 1$. Ker je bila $\lambda$ poljubna lastna vrednost, mora to veljati tudi za spektralni radij $\rho(Q)$ in po lemi \ref{matrike} je potem matrika $(I-Q)$ obrnljiva.

\item[b)] Za absorbirajoče stanje $j$ velja
$$v_{uj} = \sum_{k} w_{uk}p_{kj},$$
saj je število žetonov, ki prispejo v $j$ enako vsoti produktov med $w_{uk}$, številom izstreljenih žetonov iz stanja $k$, in $p_{kj}$, deležem žetonov, ki gredo pri izstrelitvi stanja $k$ v stanje $j$. \newline
Naj bo $\overline{v}_{uj} = \frac{v_{uj}}{v_{u}}$. Potem je
$$ \overline{v}_{uj} = \sum_{k} \overline{w}_{uk}p_{kj}.$$
Če to naredimo za vse izbire začetnega stanja $u$, dobimo matrično enačbo
$$\overline{V} = \overline{W}R,$$
kjer množimo z matriko $R$, saj gledamo verjetnosti prehodov iz minljivih v absorbirajoča stanja.
Ker pa je $\overline{W} = N$, dobimo 
$$\overline{V} = NR = B $$
Zadnja enakost velja, ker je $b_{ij} = \sum_{k\in I} n_{ik}p_{kj}$ za $i,k \in I$, $j$ absorbirajoče.

\hfill \QED
\end{enumerate}









\end{document}