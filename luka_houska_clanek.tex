\documentclass[twoside,11pt]{article}
\usepackage[slovene]{babel}
\usepackage[utf8]{inputenc}
\usepackage{graphicx}
\usepackage[frame]{matrika}
\usepackage{mathtools}
\usepackage{epstopdf}
\usepackage{units}
\usepackage{url}
\usepackage{amsfonts}


\begin{document}

\MAT{1}{10}{2024}
\naslov{Stohastični abakus}

\avtor{Luka Houška}

\institucija{Fakulteta za matematiko in fiziko \\ Univerza v Ljubljani}

\date{23.10.2023}

\klasifikacija{~} 
\izvlecek{V članku je podrobno predstavljen stohastični abakus, kako algoritem deluje, zakaj deluje in zakaj se ustavi. S pomočjo stohastičnega abakusa so analizirane igre "podajanja evra" na različnih grafih, kar razkrije naravno povezavo s Fibonaccijevimi števili.}
\title{Probabilistic abacus}
\abstract{The article presents stochastic abacus in detail, how the algorithm works, why it works and why it stops. Using the stochastic abacus, pass-the-buck games are analyzed on various graphs, revealing a natural connection with the    Fibonacci numbers.}

\glava
\baselineskip=14.5pt

\smallskip

\section{Uvod}
Igra podajanja evra se začne z $n$ igralci postavljenimi kot vozlišča grafa. En igralec ima v roki evro. Igra se igra izmenično, kjer v vsakem koraku igralec naključno zmaga ali pa poda evro enemu od igralcev na sosednjih vozliščih. Če ima vozlišče trenutnega igralca stopnjo $d$ je vsak od $d + 1$ izidov enako verjeten. Pri igri bomo uporabljali koncept chip firing, katerega je kot sredstvo za poučevanje verjetnosti osnovnošolcem razvil oče stohastičnega abakusa Arthur Engel.
Osnovna ideja chip firinga je preprosta. Če imamo podan graf, si predstavljamo, da na vozlišča postavimo nekaj žetonov. Če je število žetonov na vozlišču $v$ večje ali enako kot stopnja $v$, potem lahko "izstrelimo" $v$ tako, da po vsaki povezavi, ki ima za eno od krajišč vozlišče $v$ pošljemo en žeton do sosednjega vozlišča. Število žetonov v vozlišču $v$ se po izstrelitvi zmanjša za stopnjo $v$, vsakemu od sosednjih vozlišč $v$ pa se število žetonov poveča za ena. 
Koncept chip firing bomo implementirali na \emph{usmerjene grafe}, kjer bo izstrelitev vozlišča $v$ pomenila, da bo po vsaki izhodni povezavi iz $v$ pošljemo en žeton. Natančneje, usmerjeni graf bo prehodni graf za absorbirajočo Markovsko verigo, ki ustreza igri podajanja evra. Naša naloga je, da natančno predstavimo Engelov originalni koncept, ki uporablja chip firing algoritem kot sredstvo za računanje verjetnosti in pričakovanih vrednosti. Naš prvi korak je spoznati Engelov algoritem.

\section{Stohastični abakus}
V začetku 1970-ih je Engel razvil svoj algoritem, ki ga je kasneje poimenoval stohastični abakus, saj premikanje žetonov po povezavah v grafu za namen računanja spominja na tradicionalni abakus.
\indent Predpostavimo, da imamo tako časovno diskretno, absorbirajočo Markovsko verigo s končnim število stanj, da so vse prehodne verjetnosti racionalne. Predstavimo jo z usmerjenim grafom, kjer so stanja vozlišča in usmerjena povezava med vozliščema $u$ in $v$ obstaja natanko tedaj, ko je verjetnost direktnega prehoda iz $u$ v $v$ pozitivna. 
Natančneje, če je $q$ najmanjši skupni imenovalec vseh prehodnih verjetnosti iz stanja $u$ in je verjetnost prehoda iz $u$ v $v$ enaka $p/q$, potem postavimo p usmerjenih povezav iz $u$ v $v$. Če to naredimo za vse sosede $u$, je izhodna stopnja $u$ enaka $q$. Zato se prehodne verjetnosti lahko razberejo iz usmerjenega grafa brez uporabe uteži na povezavah. 
Uporabljali bomo izraza \emph{notranja vozlišča} za povrnljiva stanja in \emph{končna vozlišča} za absorbirajoča stanja. \\
\indent Ideja algortima je, da začnemo s praznimi vozlišči in dodajamo žetone na začetno stanje, ki ga bomo poimenovali začetno vozlišče. Ko je v začetnem vozlišču dovolj žetonov, ga izstrelimo tako, da po povezavah do sosednjih vozlišč pošljemo žetone v razmerju, ki odražajo prehodne verjetnosti. Ponavljamo postopek dodajanja žetonov v začetno vozlišče in izstreljevanja dokler nima vsak od sosedov dovolj žetonov, da izstrelimo te itn. Predstavljajmo si, da ponavljamo postopek dodajanja žetonov in izstrelitev vozlišč in
da je na neki točki število žetonov na notranjih vozliščih enako začetnemu. Engel je utemeljil, da je v tem trenutku število žetonov v končnih vozliščih proporcionalno njenim verjetnostim, saj premikanje žetonov po grafu upošteva prehodne verjetnosti. Če smo bolj natančni: če ima končno vozlišče $k$ žetonov in je število vseh dodanih žetonov v začetno vozlišče $c$, potem je verjetnost, da pridemo v to absorbirajoče stanje iz začetnega stanja enaka $k/c$. Ker smo ustvarili periodičen proces; če dodajamo žetone v začetno vozlišče in ponovimo celoten postopek, bo končna razporeditev žetonov v končnih vozliščih enaka, saj more po zakonu o velikih številih biti v razmerju z verjetnostmi. Ker pa je proces periodičen je ena iteracija dovolj, da odkrijemo pravilna razmerja žetonov v končnih vozliščih. \\
\indent S tem algoritmom sta dva potencialna problema in Engel je uspel razrešiti oba. Prvič, ni težko pokazati, da je vrstni red izstrelitev vozlišč nepomemben. To pomeni, da je "sproščanje" sistema, izstrelitev vseh primernih vozlišč dobro definirano. Drugič, treba je ugotoviti kako prvotno postaviti žetone na notranjih vozliščih, da imamo zagotovljeno, da se bo sistem vrnil v to razporeditev. Izkaže se, da če na začetku vsako notranje vozlišče "kritično naložimo", tj. ima en žeton manj kot ima izhodnih povezav, dosežemo, da se bodo notranja vozlišča vrnila v "kritično naloženo" stanje po končnem številu dodanih žetonov v začetno vozlišče. Podajmo še dokaz.

{\em Dokaz:\/} Zapišimo prehodne verjetnosti kot $p_{ij} = \frac{r_{ij}}{r_{i}},\; r_{ij}, r_{i}\in \mathbb{N}$. 






\end{document}